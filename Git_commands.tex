\documentclass[a4paper,12pt]{report}

\usepackage{alltt, fancyvrb, url}
\usepackage{graphicx}
\usepackage[utf8]{inputenc}
\usepackage{hyperref}

% Questo commentalo se vuoi scrivere in inglese.
\usepackage[italian]{babel}

\usepackage[italian]{cleveref}

\title{Guida rapida per i comandi Git}

\author{Luca Morlino}
\date{\today}
\begin{document}

\maketitle


\chapter{Operazioni preliminari}
\begin{itemize}
\item Per impostare un nome utente di default : \\\\
\textbf{git config -{}-global\footnote{il flag -{}-global setta le impostazioni non per il repository corrente ma per qualunque nuovo repository} user.name "YOUR NAME"} \\
\centerline{es. git config -{}-global user.name "JOHN SMITH"}
%
\item Per specificare una mail di default:\\\\
\textbf{git config -{}-global user.mail "your.email@provider"} \\
\centerline{es. git config -{}-global user.email johnsmith@example.com}
%
\item Impostazione carattere di new line: \\\\
Dato l'uso di diversi sistemi operativi bisogna configurare git affinchè non reinterpreti il fine linea. Il fine linea va settato sull'IDE (Text File Encoding:UTF-8; New Text File Line Delimiter: Unix) mentre su git va disattivato.\\\\
\centerline{Per utenti Unix e Mac-OS:}\\\\
\textbf{git config -{}-global core.autocrlf input}\\\\
\centerline{Per utenti Windows:}\\\\
\textbf{git config -{}-global core.autocrlf false}
\end{itemize}

\chapter{Primi passi: creare un repository e fare un commit}
Spostarsi sulla cartella dove si vuole creare il repository e con il comando
\\\\
\textbf{git init}
\\\\
verrà creata una cartella .git (inizia con il punto poichè su Unix tali cartelle sono nascoste).
\\\\
Verificare di essere sul branch master con il comando
\\\\
\textbf{git status}
\\\\
Cosa \textbf{VA} tracciato con git:
\begin{itemize}
\item Sorgenti
\item Risorse
\item Librerie
\item File esterni quali README.md, file per la licenza, il file .project per facilitare un lavoro di un team che utilizza Eclipse
\end{itemize}
Cosa \textbf{NON VA}  tracciato con git:
\begin{itemize}
\item Binari
\item Documentazione rigenerabile dai sorgenti
\item Archivi rigenerabili 
\end{itemize}
Per non tracciare file e/o cartelle creare il file .gitignore con all'interno un elenco:
\\\\
bin/
\\
doc/
\\
*.log
\\
*.pdf
\\\\
Per aggiungere le modifiche fatte su un file (compresa la cancellazione del file stesso) allo staging area lanciare:
\\\\
\textbf{git add PATH\textunderscore TO\textunderscore FILE}
\\\\
Nel caso di più file:
\\\\
\textbf{git add PATH\textunderscore TO\textunderscore FILE\textunderscore 1 PATH\textunderscore TO\textunderscore FILE\textunderscore 2 PATH\textunderscore TO\textunderscore FILE\textunderscore 3}
\\\\
Se ci si accorge di avere fatto un errore e di voler togliere uno o più file dallo stage usare il comando:
\\\\
\textbf{git reset PATH\textunderscore TO\textunderscore FILE}
\\\\
Nel caso di più file:
\\\\
\textbf{git reset PATH\textunderscore TO\textunderscore FILE\textunderscore 1 PATH\textunderscore TO\textunderscore FILE\textunderscore 2 PATH\textunderscore TO\textunderscore FILE\textunderscore 3}
\\\\
Una volta messo il file sulla staging area si potrà fare il commit:
\\\\
\textbf{git commit -m "QUI INSERIRE IL TESTO DEL COMMIT"}
\\\\
Riverificare lo stato del repository:
\\\\
\textbf{git status}
\\\\
\textbf{GOOD PRACTICES}:
\begin{itemize}
\item Molti commit
\item Piccola dimensione
\item Messaggio breve ma significativo
\item Verificare continuamente lo stato del repository
\end{itemize}
\textbf{Come visualizzare la storia dei commit}:
\\\\
Per visualizzare tutti i commit della linea di sviluppo corrente:
\\\\
\textbf{git log}
\\\\
Per visualizzare tutti i commit della linea di sviluppo corrente ma con una visualizzazione grafica:
\\\\
\textbf{git log -{}-graph}
\\\\
Per visualizzare tutti i commit di tutti i branch con una visualizzazione grafica:
\\\\
\textbf{git log -{}-all -{}-graph}

\chapter{Navigare nel repository}
Se vogliamo vedere le modifiche intercorse fra due commit usiamo il comando diff. Per riferirci ad un commit usiamo il suo codice hash (bastano le prime 7 cifre. Per ottenerlo cercarlo nello storico con git log)
\\\\
Per mostrare le differenze fra il working tree e l'ultimo commit:
\\\\
\textbf{git diff}
\\\\
Per mostrare le differenze fra il working tree e un commit specifico:
\\\\
\textbf{git diff FROM}
\\\\
\centerline{es. git diff 07388467}
\\\\
Per mostrare le differenze fra due commit specifici:
\\\\
\textbf{git diff FROM TO}
\\\\
\centerline{es. git diff 0738847 7736421}
\\\\
Per semplificare l'accesso agli ultimi commit possiamo riferirci ad essi usando HEAD $\sim$N dove N è il numero quanti  commit vogliamo spostarci all'indietro. Per esempio se vogliamo spostarci indietro di 3 commit lanciamo il comando:
\\\\
\textbf{git diff HEAD$\sim$3}
\\\\\\\\
Per spostarsi invece fisicamente su di un commit si usa il comando 

checkout:
\\\\
\textbf{git checkout COMMITREF}
\\\\
Dove COMMITREF può essere:
\begin{itemize}
\item L'hash completo di un commit
\item Almeno le prime 7 cifre di un commit
\item HEAD$\sim$2
\item master
\item il nome di un altro branch (ciò fa sì che ci si sposta alla fine di quel branch)
\end{itemize}
A quel punto si è in una modalità chiamata Detached Head (testa staccata). Ogni commit fatto in quel punto verrà scartato. Per tornare in modalità attached bisogna tornare alla HEAD del branch:
\\\\
\centerline{es. git checkout master}
\\\\
Con il comando checkout si possono anche recuperare le modifiche solo per alcuni file a mia scelta fatte in un certo commit:
\\\\
\textbf{git checkout COMMITREF -{}- FILENAME}
\\\\
Se per esempio voglio ripristinare lo stato del solo file pippo.txt allo stato che aveva 2 commit fa:
\\\\
\centerline{git checkout HEAD$\sim$2 -{}- /pippo.txt}
\\\\
\chapter{VALE GAIO}
Il sottocomando checkout consente anche di creare un nuovo branch dal punto in cui ci si trova e di spostarcisi automaticamente:
\\\\
\textbf{git checkout -b NEWBRANCHNAME}
\\\\
I commit seguenti saranno perciò aggiunti al nuovo branch.
\\
Per visualizzare tutti i branch lanciare:
\\\\
\textbf{git branch}
\\\\
La HEAD identifica la posizione corrente all'interno della storia del repository. In una normale sessione la HEAD è posizionata alla fine del branch.
Introduciamo il concetto di fusione di due branch. Il comando per ottenere ciò è merge.

Bisogna prima spostarsi sul branch in cui voler introdurre le modifiche e poi usare il comando:
\\\\
\textbf{git merge BRANCHNAME}
\\\\
Dove BRANCHNAME è il nome del branch che si vuole unire al branch su cui si è attualmente

Se non ci sono conflitti tutti i commit di BRANCHNAME vengono aggiunti al branch corrente.

Viene creato un nuovo commit di default (che è buona norma non modificare) con l'editor di default.
\\\\
Eventualmente si può ora cancellare il branch "secondario" con:
\\\\
\textbf{git branch -d BRANCHNAME}
\\\\
Attenzione: viene cancellata solo l'associazione del nome a quel branch. Nello storico il branch è ancora visibile. Ora possiamo dare lo stesso nome ad un nuovo branch.
\\\\
\section{Risolvere i merge conflict}
Nel caso in cui due linee di sviluppo abbiano modificato concorrentemente un file nello stesso punto, il merge non è banale da effettuare ma occorre risolvere manualmente un merge conflict.
Una volta modificati i file che confliggono occorre aggiungerli allo stage e poi fare il commit (il commit di default viene auto-generato).
\\\\
\textbf{Ricordarsi di tracciare le cose corrette. Non i file rigenerabili, in quanto provocano merge conflict difficili da risolvere.}


\chapter{Git in rete}

\section{Clonare un repository}
Per importare una copia di un repository esistente lanciare il comando:
\\\\
\textbf{git clone URI localfolder}
\\\\
Il comando scarica l'intera storia del repository conservato in URI all'interno di localfolder(che deve essere vuoto) che diventa un nuovo repository Git. Se localfolder non viene segnalato allora Git crea una cartella con l'ultimo nome del path.
\section{Repository remoto}
E' utile ricordare un indirizzo con un nome simbolico. 
Per conoscere i repository remoti configurati usare il comando:
\\\\
\textbf{git remote -v}
\\\\
Per aggiungerne uno nuovo:
\\\\
\textbf{git remote add name url}
\\\\
\centerline{es. git remote add serverAziendale www.github.com/pippo/repo}
\\\\
Per cancellarne uno esistente:
\\\\
\textbf{git remote rm name}
\\\\
Ogni branch sulla nostra macchina può essere configurato per avere un upstream. In pratica il branch locale è la copia del branch di un repository remoto.
Per settare ciò usare il comando:
\\\\
\textbf{git branch -u remoteName/remoteBranch}
\\\\
D'ora in poi il branch locale corrente fara' push e pull dal Branch remoteBranch.
\\\\
Nel caso il repository sia stato clonato e non inizializzato un riferimento al repository remoto di origine viene automaticamente inserito e configurato come upstream per tutti i branch col nome di origin.
\\\\
Al momento del clone Git scaricherà solo il branch primario.
Per sapere quali sono tutti i branch in remoto usare il comando:
\\\\
\textbf{git branch -a}
\\\\
Qualora si voglia importare in locale un branch remoto per lavorarci occorre creare un "tracking branch" locale:
\\\\
\textbf{git checkout -b localBranchName remoteName/remoteBranchName}
\\\\
\centerline{es. git checkout -b miaLineaSviluppo origin/developBranch}
\section{Aggiornamento dei nuovi commit}
Vogliamo sapere se su un certo branch ci sono dei nuovi commit fatti da altri utenti.
\begin{itemize}


\item Il sottocomando fetch ci permette di aggiornare le informazioni:
\\\\
\textbf{git fetch URI branchName}
\\\\
oppure
\\\\
\textbf{git fetch remoteName branchName}
\\\\
oppure qualora ci sia un branch upstream configurato
\\\\
\textbf{git fetch}
\\\\
Dopo aver fatto la fetch occorre fare la merge con "git merge" e risolvere gli eventuali conflitti.
\\
\item Il sottocomando:
\\\\
\textbf{git pull}
\\\\
permette di fare fetch e merge in un colpo solo.
Quindi perchè fare la fetch? Qualora si volesse vedere gli aggiornamenti di un branch senza effettuare poi il merge.

La sintassi è la stessa del comando fetch.
\\\\
\textbf{git pull URI branchName}
\\\\
oppure
\\\\
\textbf{git pull remoteName branchName}
\\\\
oppure qualora ci sia un branch upstream configurato
\\\\
\textbf{git pull}
\\\\
\\\\
\item Il comando speculare a pull è push. Permette di "spingere" i nuovi commit su un branch se non ci sono conflitti.

La sintassi è la stessa del comando pull. La push va a buon fine solo se c'è corrispondenza tra tutti i commmit sul locale e sul remoto. 
\\\\
\textbf{git push URI branchName}
\\\\
oppure
\\\\
\textbf{git push remoteName branchName}
\\\\
oppure qualora ci sia un branch upstream configurato
\\\\
\textbf{git push}
\\\\
\end{itemize}
\end{document}