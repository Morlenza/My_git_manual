\documentclass[a4paper,12pt]{report}

\usepackage{alltt, fancyvrb, url}
\usepackage{graphicx}
\usepackage[utf8]{inputenc}
\usepackage{hyperref}

% Questo commentalo se vuoi scrivere in inglese.
\usepackage[italian]{babel}

\usepackage[italian]{cleveref}

\title{Guida rapida per i comandi Git}

\author{Luca Morlino}
\date{\today}
\begin{document}

\maketitle


\chapter{Operazioni preliminari}
\begin{itemize}
\item Per impostare un nome utente di default : \\\\
\textbf{git config -{}-global\footnote{il flag -{}-global setta le impostazioni non per il repository corrente ma per qualunque nuovo repository} user.name "YOUR NAME"} \\
\centerline{es. git config -{}-global user.name "JOHN SMITH"}
%
\item Per specificare una mail di default:\\\\
\textbf{git config -{}-global user.mail "your.email@provider"} \\
\centerline{es. git config -{}-global user.email johnsmith@example.com}
%
\item Impostazione carattere di new line: \\\\
Dato l'uso di diversi sistemi operativi bisogna configurare git affinchè non reinterpreti il fine linea. Il fine linea va settato sull'IDE (Text File Encoding:UTF-8; New Text File Line Delimiter: Unix) mentre su git va disattivato.\\\\
\centerline{Per utenti Unix e Mac-OS:}\\\\
\textbf{git config -{}-global core.autocrlf input}\\\\
\centerline{Per utenti Windows:}\\\\
\textbf{git config -{}-global core.autocrlf false}
\end{itemize}

\chapter{Primi passi: creare un repository e fare un commit}
Spostarsi sulla cartella dove si vuole creare il repository e con il comando
\\\\
\textbf{git init}
\\\\
verrà creata una cartella .git (inizia con il punto poichè su Unix tali cartelle sono nascoste).
\\\\
Verificare di essere sul branch master con il comando
\\\\
\textbf{git status}
\\\\
Cosa \textbf{VA} tracciato con git:
\begin{itemize}
\item Sorgenti
\item Risorse
\item Librerie
\item File esterni quali README.md, file per la licenza, il file .project per facilitare un lavoro di un team che utilizza Eclipse
\end{itemize}
Cosa \textbf{NON VA}  tracciato con git:
\begin{itemize}
\item Binari
\item Documentazione rigenerabile dai sorgenti
\item Archivi rigenerabili 
\end{itemize}
Per non tracciare file e/o cartelle creare il file .gitignore con all'interno un elenco:
\\\\
bin/
\\
doc/
\\
*.log
\\
*.pdf
\\\\
Per aggiungere le modifiche fatte su un file (compresa la cancellazione del file stesso) allo staging area lanciare:
\\\\
\textbf{git add PATH\textunderscore TO\textunderscore FILE}
\\\\
Nel caso di più file:
\\\\
\textbf{git add PATH\textunderscore TO\textunderscore FILE\textunderscore 1 PATH\textunderscore TO\textunderscore FILE\textunderscore 2 PATH\textunderscore TO\textunderscore FILE\textunderscore 3}
\\\\
Se ci si accorge di avere fatto un errore e di voler togliere uno o più file dallo stage usare il comando:
\\\\
\textbf{git reset PATH\textunderscore TO\textunderscore FILE}
\\\\
Nel caso di più file:
\\\\
\textbf{git reset PATH\textunderscore TO\textunderscore FILE\textunderscore 1 PATH\textunderscore TO\textunderscore FILE\textunderscore 2 PATH\textunderscore TO\textunderscore FILE\textunderscore 3}
\\\\
Una volta messo il file sulla staging area si potrà fare il commit:
\\\\
\textbf{git commit -m "QUI INSERIRE IL TESTO DEL COMMIT"}
\\\\
Riverificare lo stato del repository:
\\\\
\textbf{git status}
\\\\
\textbf{GOOD PRACTICES}:
\begin{itemize}
\item Molti commit
\item Piccola dimensione
\item Messaggio breve ma significativo
\item Verificare continuamente lo stato del repository
\end{itemize}
\textbf{Come visualizzare la storia dei commit}:
\\\\
Per visualizzare tutti i commit della linea di sviluppo corrente:
\\\\
\textbf{git log}
\\\\
Per visualizzare tutti i commit della linea di sviluppo corrente ma con una visualizzazione grafica:
\\\\
\textbf{git log -{}-graph}
\\\\
Per visualizzare tutti i commit di tutti i branch con una visualizzazione grafica:
\\\\
\textbf{git log -{}-all -{}-graph}

\chapter{Navigare nel repository}
Se vogliamo vedere le modifiche intercorse fra due commit usiamo il comando diff. Per riferirci ad un commit usiamo il suo codice hash (bastano le prime 7 cifre. Per ottenerlo cercarlo nello storico con git log)
\\\\
Per mostrare le differenze fra il working tree e l'ultimo commit:
\\\\
\textbf{git diff}
\\\\
Per mostrare le differenze fra il working tree e un commit specifico:
\\\\
\textbf{git diff FROM}
\\\\
\centerline{es. git diff 07388467}
\\\\
Per mostrare le differenze fra due commit specifici:
\\\\
\textbf{git diff FROM TO}
\\\\
\centerline{es. git diff 0738847 7736421}
\\\\
Per semplificare l'accesso agli ultimi commit possiamo riferirci ad essi usando HEAD $\sim$N dove N è il numero quanti  commit vogliamo spostarci all'indietro. Per esempio se vogliamo spostarci indietro di 3 commit lanciamo il comando:
\\\\
\textbf{git diff HEAD$\sim$3}
\\\\
Per spostarsi invece fisicamente su di un commit si usa il comando checkout:
\\\\
\textbf{git checkout COMMITREF}
\\\\
Dove COMMITREF può essere:
\begin{itemize}
\item L'hash completo di un commit
\item Almeno le prime 7 cifre di un commit
\item HEAD$\sim$2
\item master
\item il nome di un altro branch (ciò fa sì che ci si sposta alla fine di quel branch)
\end{itemize}
A quel punto si è in una modalità chiamata Detached Head (testa staccata). Ogni commit fatto in quel punto verrà scartato. Per tornare in modalità attached bisogna tornare alla HEAD del branch:
\\\\
\centerline{es. git checkout master}
\\\\
Con il comando checkout si possono anche recuperare le modifiche solo per alcuni file a mia scelta fatte in un certo commit:
\\\\
\textbf{git checkout COMMITREF -{}- FILENAME}
\\\\
Se per esempio voglio ripristinare lo stato del solo file pippo.txt allo stato che aveva 2 commit fa:
\\\\
\centerline{git checkout HEAD$\sim$2 -{}- /pippo.txt}
\\\\
\chapter{Creare un nuovo branch e unire due branch}
Il sottocomando checkout consente anche di creare un nuovo branch e di spostarcisi automaticamente:
\\\\
\textbf{git checkout -b NEWBRANCHNAME}
\\\\
I commit seguenti saranno perciò aggiunti al nuovo branch.
\\
Per visualizzare tutti i branch lanciare:
\\\\
\textbf{git branch}
\\\\
La HEAD identifica la posizione corrente all'interno della storia del repository. In una normale sessione la HEAD è posizionata alla fine del branch.
Introduciamo il concetto di fusione di due branch. Il comando per ottenere ciò è merge.

Bisogna prima spostarsi sul branch in cui voler introdurre le modifiche e poi usare il comando:
\\\\
\textbf{git merge BRANCHNAME}
\\\\
Dove BRANCHNAME è il nome del branch che si vuole unire al branch su cui si è attualmente

Se non ci sono conflitti tutti i commit di BRANCHNAME vengono aggiunti al branch corrente.

Viene creato un nuovo commit di default (che è buona norma non modificare) con l'editor di default.
\\\\
Eventualmente si può ora cancellare il branch "secondario" con:
\\\\
\textbf{git branch -d BRANCHNAME}
\\\\
Attenzione: viene cancellata solo l'associazione del nome a quel branch. Nello storico il branch è ancora visibile. Ora possiamo dare lo stesso nome ad un nuovo branch.

\end{document}